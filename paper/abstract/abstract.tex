%\addcontentsline{toc}{chapter}{Abstract}

\begin{abstract}
	
Due to the widespread usage and amount of attacks on computer networks, a fast and accurate method to detect attacks on these networks is an ever growing need. Due to their ability to learn and to process data quickly, an Artificial Neural Network (ANN) is an increasingly popular tool for this job. A lot of research has gone into either misuse detection where signatures of known intrusion techniques are found, or simple anomaly detection using labeled data sets. These methods are however not full solutions. New and unknown attacks can not be detected and large labeled data sets require experts to carefully examine every row, leading to very few of them being available.
	
In this thesis a Recurrent Neural Network (RNN) is applied to unlabeled data, attempting to model a user's behavior and to find any deviations from this model. As RNNs have the ability to process large sequences of data and keep previous data in memory, modeling a user's behavior over longer time series and finding anomalies that span multiple actions (also known as collective anomalies) should be possible as well. In order to accomplish this, all of a user's actions in a training set are used to teach an RNN to predict a user's actions, after which the RNN predicts the user's next action. Actions that deviate from the predicted action can then be labeled as anomalous actions. Findings indicate that using an RNN for this task is technologically possible, requiring a lot of resources but being able to handle a big network (~12000 users) in real-time. However, determining if the detected anomalies are all attacks, if some may have gone undetected and what the optimal features that can be taken from the data set are, is something that cannot be determined as the data set is unlabeled. This leads to the conclusion that more research on the area of using ANNs on unlabeled data is needed.
	
	\end{abstract}