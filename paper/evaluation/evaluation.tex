\chapter{Evaluation}\label{ch:evaluation}

There are three major stages to the evaluation (preprocessing, training/testing, translating) as explained in Chapter~\ref{ch:methods}. These three stages each have different bottlenecks. Because of this not all experiments were run on the same computer. Both the preprocessing and translating stages require the reading of the entire data set file. The whole file needs to be read because it has to be sorted by individual users before being able to output the features for a percentage of those users. This makes it a very RAM intensive task, requiring around 1TB of RAM. Preprocessing and translating only require CPU work after loading in the file, shifting the bottleneck to the CPU after loading it. The preprocessing stage allows for CPU parallelization, making more CPUs an easy way to improve performance. Because of the high RAM requirement, the first and third stages are run on a computer with 1.5TB of ram and 16 Intel Xeon E5--2630v3 CPUs running at 2.40GHz with 32 threads. Because the output of the preprocessing stage is significantly smaller than the data set (about 93\% smaller), using a computer with less RAM for the training/testing stage is possible. Because of this the second task was ran on a computer with 1TB of ram, 20 Intel Xeon E5--2650v3 CPUs running at 2.30GHz with 40 threads and 8 dual-gpu boards containing two NVIDIA Tesla K80 GPUs each with 11.5GB of memory. This stage can either be very CPU- or GPU-intensive depending on which method is chosen.

In order to get an idea how long running everything on 100\% of the data set would take, two different percentages have been used: 0.1\% and 1\%. Doing preprocessing took 2h45m14s for 0.1\% of the data while it took 4h57m for 1\% of the data, both using 10 CPUs and including the reading of the file (at around 2h15m). Scaling this up linearly would put the duration of preprocessing the entire data set at about 250 hours, also using 10 CPUs. Doing the training/testing stage with 16 GPUs took 43h14m53s on 0.1\% of the data set, while 1\% took 172h25m. Since 1\% of the dataset contains 126,322,330 feature vectors, and 100\% of the data set should contain 1,051,430,459 rows, doing training/testing for 100\% of the data set should take approximately 1435h5m30s using GPUs. Using 20 CPUs instead takes about 152h35m on 1\% of the data set. This means that on in our scenario running the system on the CPU is approximately 12\% faster. Results may vary based on the number and architecture of CPUs or GPUs the system is executed on, making either CPUs or GPUs the faster option. The anomaly translation part generally only takes roughly 5 and a half hours, not varying much between data set sizes as all users need to be iterated through regardless and no other heavy CPU work is being done. A significant amount of time is spent on loading the data set file again, taking around 2h15m as well. Adding all of these times together leads to the following results. The entire process takes 48h30m7s for 0.1\% of the data set, 179h52m for 1\% of the data set and approximately 1617h23m40s for 100\% of the data set. See table~\ref{tab:times_taken} for an overview. 1\% of the data set contains 126,322,330 feature vectors, meaning the network can handle about 198 rows per second on GPUs and 221 rows per second on CPUs, making this a very good fit for real-time anomaly detection. The actual testing stage (without training) takes even shorter, generally taking about 1/100th of the time the training stage took for that user, which would make a network that does not continue learning after the initial training even more feasible to run.

\begin{table}[]
	\centering
	\caption{The time every stage takes, by data set size}
	\label{tab:times_taken}
	\begin{tabular}{llll}
										  & 0.1\%     & 1\%     & 100\%             \\ \cline{2-4} 
	\multicolumn{1}{l|}{Preprocessing}    & 2h45m14s  & 4h57m   & \(\sim\)250h        \\
	\multicolumn{1}{l|}{Training/Testing} & 43h14m53s & 172h25m & \(\sim\)1435h5m30s \\
	\multicolumn{1}{l|}{Translating}      & 5h20m     & 5h20m   & 5h20m             \\
	\multicolumn{1}{l|}{Total}            & 51h20m7s  & 182h42m & \(\sim\)1690h25m30s
	\end{tabular}
\end{table}