\chapter{Introduction}  \label{ch:introduction}
In this chapter we give an introduction to the problem addressed in this thesis.


\section{Applications}
Chapters may include sections.

To make sure that this document renders correctly, execute these commands:
\begin{quote}
\begin{verbatim}
pdflatex thesis
bibtex thesis
pdflatex thesis
pdflatex thesis
\end{verbatim}
\end{quote}
Here, the \verb|pdflatex| command may need to be executed three times in order to generate the table of contents and so on. 
Note that a good thesis has figures and tables; examples can be found in Figure~\ref{fig:afigure} and Table~\ref{tab:atable}. And every thesis has references, like~\cite{brilliantgift15}.

\begin{figure}
\begin{center}
\end{center}
\caption{Every thesis should have figures.\label{fig:afigure}}
\end{figure}

\begin{table}
\begin{center}
\begin{tabular}{ll}
Column A & Column B\\
\hline
Point 1 & Good\\
Point 2 & Bad\\
\end{tabular}
\end{center}
\caption{Every thesis should have tables.\label{tab:atable}}
\end{table}

Final reminder: this template is just an example, if you want you can make adjustments; also discuss with your supervisor which layout he or she likes. But the front page should be as it is now.

TODO: quite a lot!

\section{Thesis Overview}
It is recommended to end the introduction with an overview of the thesis. This chapter contains the introduction; Chapter~\ref{ch:definitions} includes the definitions; Chapter~\ref{ch:relatedwork} discusses related work; Chapter~\ref{ch:evaluation} evaluates the contributions; Chapter~\ref{ch:conclusions} concludes.

Also make a nice sentence with ``bachelor thesis'', LIACS and the names of the supervisors.

