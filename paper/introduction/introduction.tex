\chapter{Introduction}\label{ch:introduction}
Due to the constantly increasing threat of digital attacks on networks, and the constantly increasing presence of networks in our day-to-day lives, defending against these attacks is becoming a more and more pressing issue. Network administrators attempt to log any activity going on on their networks, hoping to be able to pick out any possible intrusions in the network. The amount of data, however is simply too big for humans to process, all while needing an expert's opinion on this data in order to spot any events that might be intrusions. This calls for a system to analyze this log data in order to find any intrusions, also known as an Intrusion Detection System (IDS). This IDS needs to be both fast and accurate, while at the same time being able to adapt to any changes the attackers might make to avoid it. The system should preferably also be able to run in real-time, being able to pick out any weird behavior as it happens, instead of finding out weeks after the fact as finding out when an intrusion happens can be very helpful knowledge when it comes to confidential data. The upcoming field of artificial neural networks (ANN) seems like a perfect fit for this problem, as it combines both the speed of computers, and attempts to mimic the ability of our brains to learn very quickly, allowing it to make good choices.

%cSpell:words akent
In 2015,~\cite{akent-2015-enterprise-data} published a dataset of around 100GB representing 58 consecutive days of de-identified event data collected from the US based Los Alamos National Laboratory's internal network. This dataset consists of a number of different types of data. These types are authentication data, process data, network flow data, DNS data and red team data, where the authentication data is by far the biggest at 1,648,275,307 events. Here the red team data represents a set of simulated intrusions. The red team data is there to train the system on known intrusions (also known as misuse detection), however there is so little red team data that it is not feasible to do this. Seeing as the rest of the data is normal (non-labeled) data, the system needs to be trained on this data, and has to try to find differences in behavior (also known as anomalies). Because this data is about series of events where differences in behavior are by definition only anomalies in the context of previous actions (also known as a collective anomaly), a recurrent neural network, which specializes in series of data, is a perfect fit.

It is recommended to end the introduction with an overview of the thesis. This chapter contains the introduction; Chapter~\ref{ch:related_work} discusses related work; Chapter~\ref{ch:conclusions} concludes.

Also make a nice sentence with ``bachelor thesis'', LIACS and the names of the supervisors.
%TODO:
